\documentclass[11pt,a4paper,oneside]{article}

\usepackage[english,russian]{babel}
\usepackage[T2A]{fontenc}
\usepackage[utf8]{inputenc}
\usepackage[russian]{olymp}
\usepackage{graphicx}
\usepackage{expdlist}
\usepackage{mfpic}
\usepackage{amsmath}
\usepackage{amssymb}
\usepackage{comment}
\usepackage{listings}
\usepackage{epigraph}
\usepackage{MnSymbol,wasysym}
\usepackage{marvosym}
\usepackage{url}
\usepackage{ulem}
\usepackage{amssymb}
\usepackage{ifsym}

\DeclareMathOperator{\nott}{not}

\begin{document}

\renewcommand{\t}[1]{\mbox{\texttt{#1}}}
\newcommand{\s}[1]{\mbox{``\t{#1}''}}
\newcommand{\eps}{\varepsilon}
\renewcommand{\phi}{\varphi}
\newcommand{\plainhat}{{\char 94}}

\newcommand{\Z}{\mathbb{Z}}
\newcommand{\w}[1]{``\t{#1}''}

\binoppenalty=10000
\relpenalty=10000

\createsection{\Note}{Комментарий}

\contest{Домашнее задание по базам данных}{Филиппов Дмитрий, М3439}{28 ноября 2016 года}

Филиппов Дмитрий, М3439
\newline

\begin{LARGE} \textbf{Домашнее задание 8.} \end{LARGE}
\newline

\textbf{Схема БД:} Flights(FlightId, FligtTime, PlaneId), Seats(PlaneId, SeatNo).

\textbf{Использованная БД:} PostgreSQL 9.4.5.
\newline

1. \textbf{Дополните эту базу данных, так что бы она поддерживала бронирование и покупку мест. При этом, бронь должна автоматически сниматься по тайм-ауту. При этом, должны поддерживаться следующие свойства:}
\begin{itemize}
\item Одно место не может быть продано или забронировано более чем один раз (в том числе, продано и забронировано одновременно).
\item Бронь можно обновить, после чего она будет действительна ещё одни сутки.
\item Бронь автоматически снимается через сутки после последнего обновления, но не позже, чем за сутки, до вылета рейса.
\item Бронирование автоматически закрывается за сутки до времени рейса.
\item Продажи автоматически закрываются не позднее двух часов до вылета рейса, либо при распродаже всех мест либо по запросу администратора.
\end{itemize}

Дополним базы данных $Airline$ следующим образом:

\begin{itemize}
  \item Добавим таблицу $Transaction$, содержающую информацию о покупке/бронировании места;
  \item А именно, в ней мы будем хранить:
    \begin{itemize}
      \item информацию о полете~--- $FlightId$, $PlaneId$, которые также будут ссылаться на таблицу $Flights$;
      \item информацию о месте~--- $PlaneId$, $SeatNo$, которые также будут ссылаться на таблицу $Seats$;
      \item информацию о клиенте, покупающем/бронирующем место: серия и номер паспорта $PassportSeries$, $PassportNo$;
      \item время и тип транзакции~--- $TransTime$, $TransType$, где $TransType$~--- либо бронирование, либо покупка.
    \end{itemize}
  \item Также нам нужно поддерживать возможность закрытия продаж по запросу администратора, для этого в таблицу $Flights$ добавим флаг
        $ClosedByRequest$.
\end{itemize}

Для снятия брони по тайм-ауту и поддержания всех вышеописанных свойств будем использовать триггеры, у нас есть вся информация для проверки всех этих условий, а значит все ок. Описание триггеров на SQL будет сделано ниже.

Все будет устроено следующим образом:

\begin{itemize}
  \item Бронь не будет автоматически сниматься по тайм-ауту (не будет listener'а, который это делает), однако просто при каждом действии мы будем проверять, не прошел ли тайм-аут с последнего бронирования (или до вылета осталось мало времени), и если прошел, считаем, что бронь снялась;
  \item Для проверки, что одно место не продано и не забронировано несколько раз проверяем при покупке/бронировании таблицу $Transactions$ и смотрим, нет ли там уже покупки/брони с неистекшим тайм-аутом на место, которое мы сейчас хотим купить;
  \item Для обновления брони опять же проверим, что она все еще действительна, если нет, будет считать, что обновить ее нельзя, надо делать новую, это другой запрос;
  \item Автоматическое снятие и закрытие брони/продажи, как писалось выше, формально мы делать не будем, однако везде будем это проверять;
  \item Закрытие при распродаже всех мест можно не обрабатывать~--- покупку отменить нельзя, поэтому уже ничего не изменится и не испортится.
\end{itemize}

2. \textbf{Запишите определение базы данных Airline на языке SQL.}

\textit{Определение таблиц и данных в них:}

\begin{lstlisting}[
           language=SQL,
           showspaces=false,
           basicstyle=\ttfamily,
           numbers=none,
           numberstyle=\tiny,
           commentstyle=\color{gray}
        ]
    DROP DATABASE IF EXISTS airline;
    CREATE DATABASE airline;
    \c airline;

    CREATE TABLE Flights (
      FlightId INT NOT NULL,
      FlightTime TIMESTAMP DEFAULT '1970-01-01 00:00:01',
      PlaneId INT NOT NULL,
      ClosedByRequest BOOLEAN DEFAULT FALSE,
      PRIMARY KEY (FlightId, PlaneId)
    );

    CREATE TABLE Seats (
      PlaneId INT NOT NULL,
      SeatNo INT NOT NULL,
      PRIMARY KEY (PlaneId, SeatNo)
    );

    CREATE TABLE Transaction (
      FlightId INT NOT NULL,
      PlaneId INT NOT NULL,
      SeatNo INT NOT NULL,
      PassportSeries INT DEFAULT NULL,
      PassportNo INT DEFAULT NULL,
      TransTime TIMESTAMP DEFAULT now(),
      TransType INT DEFAULT 0,
      //0 - booking, 1 - bying
      //comment should be defined with two dashes,
      //but I have troubles with latex then :( 
      FOREIGN KEY (FlightId, PlaneId) REFERENCES Flights,
      FOREIGN KEY (PlaneId, SeatNo) REFERENCES Seats
      ON DELETE CASCADE
      ON UPDATE CASCADE,
      PRIMARY KEY (FlightId, PlaneId, SeatNo)
    );
\end{lstlisting}

\textbf{Триггеры для проверки условий:}

\textit{Функция для проверки, что бронь больше недоступна:}
\newline

\begin{lstlisting}[
           language=SQL,
           showspaces=false,
           basicstyle=\ttfamily,
           numbers=none,
           numberstyle=\tiny,
           commentstyle=\color{gray}
        ]
    CREATE FUNCTION isBookingExpired(flightId INT, planeId INT, seatNo INT)
    RETURNS BOOLEAN AS $BODY$
      DECLARE flightTime TIMESTAMP;
      DECLARE lastBookingUpdate TIMESTAMP;
      DECLARE isClosedByRequest BOOLEAN DEFAULT FALSE;
    BEGIN
      SELECT (F.FlightTime, F.ClosedByRequest) INTO
        flightTime, isClosedByRequest
      FROM
        Flights as F
      WHERE
        F.FlightId = flightId AND
        F.PlaneId = planeId;

      SELECT T.TransTime INTO
        lastBookingUpdate
      FROM
        Transaction as T
      WHERE
        T.FlightId = flightId AND
        T.PlaneId = planeId AND
        T.SeatNo = seatNo AND
        T.TransType = 0;

      RETURN isClosedByRequest OR
             (lastBookingUpdate IS NOT NULL AND
             lastBookingUpdate + INTERVAL '24 hours' < NOW()) OR
             NOW() + INTERVAL '24 hours' > flightTime;
    END;
    $BODY$ LANGUAGE plpgsql;
\end{lstlisting}

\textit{Функция для обновления текущей брони (если бронь больше не доступна, она удаляется):}

\begin{lstlisting}[
           language=SQL,
           showspaces=false,
           basicstyle=\ttfamily,
           numbers=none,
           numberstyle=\tiny,
           commentstyle=\color{gray}
        ]
    CREATE FUNCTION updateBooking()
    RETURNS TRIGGER AS $BODY$
      DECLARE flightId INT;
      DECLARE planeId INT;
      DECLARE seatNo INT;
    BEGIN
      flightId := NEW.FlightId;
      planeId := NEW.PlaneId;
      seatNo := NEW.SeatNo;
      IF isBookingExpired(flightId, planeId, seatNo) THEN
        //Booking is expired, we cannot update it
        DELETE FROM
          Transaction as T
        WHERE
          T.FlightId = flightId AND
          T.PlaneId = planeId AND
          T.SeatNo = seatNo AND
          T.TransType = 0;
        RETURN NEW;
      ELSE
        UPDATE
          Transaction as T
        SET 
          T.TransTime = now()
        WHERE
          T.FlightId = flightId AND
          T.PlaneId = planeId AND
          T.SeatNo = seatNo AND
          T.TransType = 0;
        RETURN NEW;
      END IF;
    END;
    $BODY$ LANGUAGE plpgsql;
\end{lstlisting}

\textit{Процедура покупки места:}

\begin{lstlisting}[
           language=SQL,
           showspaces=false,
           basicstyle=\ttfamily,
           numbers=none,
           numberstyle=\tiny,
           commentstyle=\color{gray}
        ]
    CREATE FUNCTION buySeat()
    RETURNS TRIGGER AS $BODY$
      DECLARE flightId INT;
      DECLARE planeId INT;
      DECLARE seatNo INT;
      DECLARE passportSeries INT;
      DECLARE passportNo INT;
      DECLARE flightTime TIMESTAMP;
      DECLARE isClosedByRequest BOOLEAN DEFAULT FALSE;
      DECLARE nowPassportSeries INT DEFAULT NULL;
      DECLARE nowPassportNo INT DEFAULT NULL;
      DECLARE transType INT DEFAULT NULL;
    BEGIN
      flightId := NEW.FlightId;
      planeId := NEW.PlaneId;
      seatNo := NEW.SeatNo;
      passportSeries := NEW.PassportSeries;
      passportNo := NEW.PassportNo;
      SELECT (F.FlightTime, F.isClosedByRequest) INTO
        flightTime, isClosedByRequest
      FROM
        Flights as F
      WHERE
        F.FlightId = flightId AND
        F.PlaneId = planeId;

      IF (NOT F.isClosedByRequest AND NOW() <=
         flightTime - INTERVAL '2 hours') THEN
        SELECT (T.PassportSeries, T.PassportNo, T.TransType) INTO
          nowPassportSeries, nowPassportNo, transType
        FROM
          Transaction as T
        WHERE
          T.FlightId = flightId AND
          T.PlaneId = planeId AND
          T.SeatNo = seatNo;
        IF (nowPassportSeries IS NULL AND 
            nowPassportNo IS NULL) OR
           (transType = 0 AND
            nowPassportSeries = passportSeries AND
            nowPassportNo = passportNo) OR
           (transType = 0 AND
            isBookingExpired(flightId, planeId, seatNo)) THEN
          DELETE FROM
            Transaction as T
          WHERE
            T.FlightId = flightId AND
            T.PlaneId = planeId AND
            T.SeatNo = seatNo;

          INSERT INTO Transaction
            (FlightId, PlaneId, SeatNo,
             PassportSeries, PassportNo, TransType)
          VALUES
            (flightId, planeId, seatNo,
             passportSeries, passportNo, 1);
          RETURN NEW;
        ELSE
          //seat it already bought or booked, can't buy it
          RETURN OLD;
        END IF;
      ELSE
        //bying is closed or it is too late
        RETURN OLD;
      END IF;
    END;
    $BODY$ LANGUAGE plpgsql;
\end{lstlisting}

\textit{Процедура бронирования места:}

\begin{lstlisting}[
           language=SQL,
           showspaces=false,
           basicstyle=\ttfamily,
           numbers=none,
           numberstyle=\tiny,
           commentstyle=\color{gray}
        ]
    CREATE FUNCTION bookSeat()
    RETURNS TRIGGER AS $BODY$
      DECLARE flightId INT;
      DECLARE planeId INT;
      DECLARE seatNo INT;
      DECLARE passportSeries INT;
      DECLARE passportNo INT;
      DECLARE flightTime TIMESTAMP;
      DECLARE isClosedByRequest BOOLEAN DEFAULT FALSE;
      DECLARE nowPassportSeries INT DEFAULT NULL;
      DECLARE nowPassportNo INT DEFAULT NULL;
      DECLARE transType INT DEFAULT NULL;
    BEGIN
      flightId := NEW.FlightId;
      planeId := NEW.PlaneId;
      seatNo := NEW.SeatNo;
      passportSeries := NEW.PassportSeries;
      passportNo := NEW.PassportNo;
      SELECT (F.FlightTime, F.isClosedByRequest) INTO
        flightTime, isClosedByRequest
      FROM
        Flights as F
      WHERE
        F.FlightId = flightId AND
        F.PlaneId = planeId;

      IF (NOT F.isClosedByRequest AND NOW() <=
          flightTime - INTERVAL '24 hours') THEN
        SELECT (T.PassportSeries, T.PassportNo, T.TransType) INTO
          nowPassportSeries, nowPassportNo, transType
        FROM
          Transaction as T
        WHERE
          T.FlightId = flightId AND
          T.PlaneId = planeId AND
          T.SeatNo = seatNo;
        IF (nowPassportSeries IS NULL AND 
            nowPassportNo IS NULL) OR
           (transType = 0 AND
            isBookingExpired(flightId, planeId, seatNo)) THEN
          DELETE FROM
            Transaction as T
          WHERE
            T.FlightId = flightId AND
            T.PlaneId = planeId AND
            T.SeatNo = seatNo;

          INSERT INTO Transaction
            (FlightId, PlaneId, SeatNo, PassportSeries, PassportNo, TransType)
          VALUES
            (flightId, planeId, seatNo, passportSeries, passportNo, 0);
          RETURN NEW;
        ELSE
          //seat it already booked
          //(maybe by yourself, but it doesn't matter), can't buy it
          RETURN OLD;
        END IF;
      ELSE
        //booking is closed or it is too late
        RETURN OLD;
      END IF;
    END;
    $BODY$ LANGUAGE plpgsql;
\end{lstlisting}

\textit{Теперь напишем триггеры для обновления базы данных:}

\begin{lstlisting}[
           language=SQL,
           showspaces=false,
           basicstyle=\ttfamily,
           numbers=none,
           numberstyle=\tiny,
           commentstyle=\color{gray}
        ]
    CREATE TRIGGER BuySeatTrigger
    AFTER INSERT OR UPDATE ON Transaction
    FOR EACH ROW EXECUTE PROCEDURE buySeat();

    CREATE TRIGGER BookSeatTrigger
    AFTER INSERT OR UPDATE ON Transaction
    FOR EACH ROW EXECUTE PROCEDURE bookSeat();

    CREATE TRIGGER UpdateBookingTrigger
    AFTER UPDATE ON Transaction
    FOR EACH ROW EXECUTE PROCEDURE updateBooking();
\end{lstlisting}

3. \textbf{Определите, какие индексы требуется добавить к таблицам базы данных Airline.}

  Добавление этих индексов представлено в пункте 5.

4. \textbf{Пусть частым запросом является определение средней заполненности самолёта по рейсу. Какие индексы могут помочь при исполнении данного запроса?}

  Хеш-индекс по $PlaneId$ в таблице $Seats$ для быстрого получения количества мест в самолете.

  Хеш-индекс по $FlightId$ в таблице $Transactions$ для быстрого получения количества человек, летящих данным рейсом.

5. \textbf{Запишите добавление индексов на языке SQL.}

  \textit{Индексы из пункта 4:}
  \begin{itemize}
    \item CREATE INDEX SeatsIndex ON Seats USING HASH (PlaneId);
    \item CREATE INDEX FlightsIndex ON Flights USING HASH (FlightId);
  \end{itemize}

  \textit{Индексы из пункта 3:}
  \begin{itemize}
    \item CREATE INDEX TransactionsIndex ON Transactions USING HASH (FlightId);
    \item CREATE INDEX TransactionsSeatIndex ON Transactions USING BTREE (FlightId, SeatNo);
    \item CREATE UNIQUE INDEX FlightsBTIndex ON Flights USING BTREE (FlightId, FlightTime);
  \end{itemize}  

\end{document}