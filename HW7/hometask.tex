\documentclass[11pt,a4paper,oneside]{article}

\usepackage[english,russian]{babel}
\usepackage[T2A]{fontenc}
\usepackage[utf8]{inputenc}
\usepackage[russian]{olymp}
\usepackage{graphicx}
\usepackage{expdlist}
\usepackage{mfpic}
\usepackage{amsmath}
\usepackage{amssymb}
\usepackage{comment}
\usepackage{listings}
\usepackage{epigraph}
\usepackage{MnSymbol,wasysym}
\usepackage{marvosym}
\usepackage{url}
\usepackage{ulem}
\usepackage{amssymb}
\usepackage{ifsym}

\DeclareMathOperator{\nott}{not}

\begin{document}

\renewcommand{\t}[1]{\mbox{\texttt{#1}}}
\newcommand{\s}[1]{\mbox{``\t{#1}''}}
\newcommand{\eps}{\varepsilon}
\renewcommand{\phi}{\varphi}
\newcommand{\plainhat}{{\char 94}}

\newcommand{\Z}{\mathbb{Z}}
\newcommand{\w}[1]{``\t{#1}''}

\binoppenalty=10000
\relpenalty=10000

\createsection{\Note}{Комментарий}

\contest{Домашнее задание по базам данных}{Филиппов Дмитрий, М3439}{21 ноября 2016 года}

Филиппов Дмитрий, М3439
\newline

\begin{LARGE} \textbf{Домашнее задание 7.} \end{LARGE}
\newline

\textbf{Схема БД:} Course(\underline{CId}, GId, CName), Lecturer(\underline{LId}, LName), Mark(\underline{SId}, \underline{CId}, GId, Mark), Groups(\underline{GId}, GName), Student(\underline{SId}, GId, SName), Plan(LId, \underline{CId}, \underline{GId}).
\newline

Запросы:

1. \textbf{Напишите запрос, удаляющий всех студентов, не имеющих долгов.}

\begin{lstlisting}[
           language=SQL,
           showspaces=false,
           basicstyle=\ttfamily,
           numbers=none,
           numberstyle=\tiny,
           commentstyle=\color{gray}
        ]
    DELETE FROM
        Student S 
    WHERE
        NOT EXISTS
        (SELECT * FROM
            Mark M 
         WHERE 
            M.SId = S.SId AND M.Mark < 60);
\end{lstlisting}

2. \textbf{Напишите запрос, удаляющий всех студентов, имеющих 3 и более долгов.}

\begin{lstlisting}[
           language=SQL,
           showspaces=false,
           basicstyle=\ttfamily,
           numbers=none,
           numberstyle=\tiny,
           commentstyle=\color{gray}
        ]
    DELETE FROM
        Student S
    WHERE
        (SELECT COUNT(*) FROM
            (SELECT * FROM
                Mark M
             WHERE 
                M.SId = S.SId AND M.Mark < 60
            ) as debts) >= 3;
\end{lstlisting}

3. \textbf{Напишите запрос, удаляющий все группы, в которых нет студентов.}

\begin{lstlisting}[
           language=SQL,
           showspaces=false,
           basicstyle=\ttfamily,
           numbers=none,
           numberstyle=\tiny,
           commentstyle=\color{gray}
        ]
    DELETE FROM
        Groups G
    WHERE
        NOT EXISTS
        (SELECT * FROM
            Student S
         WHERE S.GId = G.GId);
\end{lstlisting}

4. \textbf{Создайте view Losers в котором для каждого студента, имеющего долги указано их количество.}
\begin{lstlisting}[
           language=SQL,
           showspaces=false,
           basicstyle=\ttfamily,
           numbers=none,
           numberstyle=\tiny,
           commentstyle=\color{gray}
        ]
    CREATE VIEW
        Losers
    AS
        SELECT
            S.SId, S.SName,
            (SELECT COUNT(*) FROM
                (SELECT * FROM
                    Mark M
                 WHERE 
                    M.SId = S.SId AND M.Mark < 60
                ) as debts_info) as debts_count
        FROM
            Student S;
\end{lstlisting}

5. \textbf{Создайте таблицу LoserT, в которой содержится та же информация, что во view Losers. Эта таблица должна автоматически обновляться при изменении таблицы с баллами.}
\begin{lstlisting}[
           language=SQL,
           showspaces=false,
           basicstyle=\ttfamily,
           numbers=none,
           numberstyle=\tiny,
           commentstyle=\color{gray}
        ]
    SELECT * INTO LoserT FROM Losers;

    CREATE FUNCTION updateLoserT()
        RETURNS TRIGGER AS $BODY$
    BEGIN
        TRUNCATE LoserT;
        INSERT INTO LoserT (SELECT * FROM Losers);
        RETURN NEW;
    END;
    $BODY$ LANGUAGE plpgsql;

    CREATE TRIGGER UpdateLoserT
    AFTER INSERT OR UPDATE ON Mark
    FOR EACH ROW EXECUTE PROCEDURE updateLoserT();
\end{lstlisting}

6. \textbf{Отключите автоматическое обновление таблицы LoserT.}
\begin{lstlisting}[
           language=SQL,
           showspaces=false,
           basicstyle=\ttfamily,
           numbers=none,
           numberstyle=\tiny,
           commentstyle=\color{gray}
        ]
    DROP TRIGGER UpdateLoserT ON Mark; 
\end{lstlisting}

или же

\begin{lstlisting}[
           language=SQL,
           showspaces=false,
           basicstyle=\ttfamily,
           numbers=none,
           numberstyle=\tiny,
           commentstyle=\color{gray}
        ]
    ALTER TABLE Mark DISABLE TRIGGER UpdateLoserT;
\end{lstlisting}

(с возможностью последующего включения с помощью \texttt{ALTER TABLE Mark ENABLE TRIGGER UpdateLoserT;})

7. \textbf{Напишите запрос (один), которой обновляет таблицу LoserT, используя данные из таблицы NewPoints, в которой содержится информация о баллах, проставленных за последний день.}

Будем считать, что таблица NewPoints содержит поля \texttt{SId INT}, \texttt{SName VARCHAR(45)}, \texttt{CId INT}, \texttt{Mark INT}.

\begin{lstlisting}[
           language=SQL,
           showspaces=false,
           basicstyle=\ttfamily,
           numbers=none,
           numberstyle=\tiny,
           commentstyle=\color{gray}
        ]
    MERGE INTO LoserT LT
    USING (SELECT * FROM NewPoints NP, Mark M) NPM
    ON LT.SId = NPM.SId AND M.CId = NPM.CId
    WHEN MATCHED THEN
        UPDATE SET LT.debts_count = 
            CASE 
                WHEN M.Mark < 60 AND NPM.Mark >= 60 THEN LT.debts_count - 1
                WHEN M.Mark >= 60 AND NPM.Mark < 60 THEN LT.debts_count + 1
                ELSE LT.debts_count
            END
    WHEN NOT MATCHED THEN
        IF NPM.Mark < 60 THEN
            INSERT VALUES (NPM.SId, NPM.SName, 1);
        ELSE
            INSERT VALUES (NPM.SID, NPM.SName, 0);
        END IF;
\end{lstlisting}

8. \textbf{Добавьте проверку того, что все студенты одной группы изучают один и тот же набор курсов.}

    В нашей схеме базы данных эта проверка не нужна, потому что нельзя добавить данные, не удовлетворяющие этому условию, в базу: таблица $Plan$ хранит $LId$, $CId$, $GId$, но не $SId$, поэтому все студенты группы $GId$ (а каждый студент привязан к своей группе) обучаются по этому плану, и все предметы у них одинаковые. При добавлении же в $Mark$ добавить данные, не удовлетворяющие условию, тоже не получится, потому что там $(CId, GId)$~--- primary key, а $(SId, GId)$~--- foreign key.

9. \textbf{Создайте триггер, не позволяющий уменьшить баллы студента по предмету. При попытке такого изменения, баллы изменяться не должны.}
\begin{lstlisting}[
           language=SQL,
           showspaces=false,
           basicstyle=\ttfamily,
           numbers=none,
           numberstyle=\tiny,
           commentstyle=\color{gray}
        ]
    CREATE FUNCTION DontAllowToDecreaseMarks()
        RETURNS TRIGGER AS $BODY$
    BEGIN
        IF NEW.Mark < OLD.Mark THEN
            RETURN OLD;
        END IF;
        RETURN NEW;
    END;
    $BODY$ LANGUAGE plpgsql;

    CREATE TRIGGER DontAllowToDecreaseMarks
    BEFORE UPDATE ON Mark
    FOR EACH ROW EXECUTE PROCEDURE DontAllowToDecreaseMarks();
\end{lstlisting}

\end{document}