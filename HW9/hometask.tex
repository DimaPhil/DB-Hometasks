\documentclass[11pt,a4paper,oneside]{article}

\usepackage[english,russian]{babel}
\usepackage[T2A]{fontenc}
\usepackage[utf8]{inputenc}
\usepackage[russian]{olymp}
\usepackage{graphicx}
\usepackage{expdlist}
\usepackage{mfpic}
\usepackage{amsmath}
\usepackage{amssymb}
\usepackage{comment}
\usepackage{listings}
\usepackage{epigraph}
\usepackage{MnSymbol,wasysym}
\usepackage{marvosym}
\usepackage{url}
\usepackage{ulem}
\usepackage{amssymb}
\usepackage{ifsym}

\DeclareMathOperator{\nott}{not}

\begin{document}

\renewcommand{\t}[1]{\mbox{\texttt{#1}}}
\newcommand{\s}[1]{\mbox{``\t{#1}''}}
\newcommand{\eps}{\varepsilon}
\renewcommand{\phi}{\varphi}
\newcommand{\plainhat}{{\char 94}}

\newcommand{\Z}{\mathbb{Z}}
\newcommand{\w}[1]{``\t{#1}''}

\binoppenalty=10000
\relpenalty=10000

\createsection{\Note}{Комментарий}

\contest{Домашнее задание по базам данных}{Филиппов Дмитрий, М3439}{5 декабря 2016 года}

Филиппов Дмитрий, М3439
\newline

\begin{LARGE} \textbf{Домашнее задание 9.} \end{LARGE}
\newline

\textbf{Схема БД:} Flights(\underline{FlightId}, FligtTime, \underline{PlaneId}, ClosedByRequest), Seats(\underline{PlaneId}, SeatNo), Transaction(\underline{FlightId}, PlaneId, \underline{SeatNo}, PassportSeries, PassportNo, TransTime, TransType).

\textbf{Использованная БД:} PostgreSQL 9.4.5.
\newline

\textbf{Реализуйте запросы к базе данных Airline с применением хранимых процедур и функций.}

Во всех процедурах учитывались условия прошлого домашнего задания (возможность закрытия регистрации администратором, невозможность бронирования за 24 часа до вылета, невозможность покупки за 2 часа до вылета).

Описание БД из прошлого дз:

\textbf{Схема изначально данной БД:} Flights(FlightId, FligtTime, PlaneId), Seats(PlaneId, SeatNo).

\begin{itemize}
  \item Добавим таблицу $Transaction$, содержающую информацию о покупке/бронировании места;
  \item А именно, в ней мы будем хранить:
    \begin{itemize}
      \item информацию о полете~--- $FlightId$, $PlaneId$, которые также будут ссылаться на таблицу $Flights$;
      \item информацию о месте~--- $PlaneId$, $SeatNo$, которые также будут ссылаться на таблицу $Seats$;
      \item время и тип транзакции~--- $TransTime$, $TransType$, где $TransType$~--- либо бронирование, либо покупка.
    \end{itemize}
  \item Также нам нужно поддерживать возможность закрытия продаж по запросу администратора, для этого в таблицу $Flights$ добавим флаг
        $ClosedByRequest$.
\end{itemize}

\textbf{Код создания БД:}

\begin{lstlisting}[
           language=SQL,
           showspaces=false,
           basicstyle=\ttfamily,
           numbers=none,
           numberstyle=\tiny,
           commentstyle=\color{gray}
        ]
    CREATE TABLE Flights (
      FlightId INT NOT NULL,
      FlightTime TIMESTAMP DEFAULT '1970-01-01 00:00:01',
      PlaneId INT NOT NULL,
      ClosedByRequest BOOLEAN DEFAULT FALSE,
      PRIMARY KEY (FlightId)
    );

    CREATE TABLE Seats (
      PlaneId INT NOT NULL,
      SeatNo INT NOT NULL,
      PRIMARY KEY (PlaneId, SeatNo)
    );
    
    CREATE TABLE Transaction (
      FlightId INT NOT NULL,
      PlaneId INT NOT NULL,
      SeatNo INT NOT NULL,
      TransTime TIMESTAMP DEFAULT now(),
      TransType INT DEFAULT 0,
      //0 - reservation, 1 - bying
      //Sorry for commenting using not double dashes
      //I'm not good in latex :(
      FOREIGN KEY (FlightId) REFERENCES Flights,
      FOREIGN KEY (PlaneId, SeatNo) REFERENCES Seats
      ON DELETE CASCADE
      ON UPDATE CASCADE,
      PRIMARY KEY (FlightId, SeatNo)
    );
\end{lstlisting}

\textit{Сначала реализуем функцию, проверяющую, что бронь истекла, которая не относится ни к одному заданию, но будет использоваться:}

\begin{lstlisting}[
           language=SQL,
           showspaces=false,
           basicstyle=\ttfamily,
           numbers=none,
           numberstyle=\tiny,
           commentstyle=\color{gray}
        ]
    CREATE FUNCTION IsReservationExpired(FlightId INT, SeatNo INT)
    RETURNS BOOLEAN AS $BODY$
      DECLARE FlightTime TIMESTAMP;
      DECLARE LastReservationUpdate TIMESTAMP;
      DECLARE IsClosedByRequest BOOLEAN DEFAULT FALSE;
    BEGIN
      SELECT (F.FlightTime, F.ClosedByRequest) INTO
        FlightTime, IsClosedByRequest
      FROM
        Flights as F
      WHERE
        F.FlightId = FlightId;

      SELECT T.TransTime INTO
        LastReservationUpdate
      FROM
        Transaction as T
      WHERE
        T.FlightId = FlightId AND
        T.SeatNo = SeatNo AND
        T.TransType = 0;

      RETURN IsClosedByRequest OR
             (LastReservationUpdate IS NOT NULL AND
             LastReservationUpdate + INTERVAL '24 hours' < NOW()) OR
             NOW() + INTERVAL '24 hours' > FlightTime;
    END;
    $BODY$ LANGUAGE plpgsql;
\end{lstlisting}

1. \textbf{FreeSeats(FlightId)~--- список мест, доступных для продажи и бронирования.}

\begin{lstlisting}[
           language=SQL,
           showspaces=false,
           basicstyle=\ttfamily,
           numbers=none,
           numberstyle=\tiny,
           commentstyle=\color{gray}
        ]
    CREATE FUNCTION FreeSeats(FlightId INT)
    RETURNS TABLE (SeatNo INT) AS $BODY$
      DECLARE PlaneId INT DEFAULT NULL;
    BEGIN
      SELECT (PlaneId) INTO
        PlaneId
      FROM 
        Flights as F
      WHERE
        F.FlightId = FlightId;

      RETURN QUERY
      SELECT
        S.SeatNo
      FROM
        Seats as S
      WHERE
        S.PlaneId = PlaneId
      EXCEPT
      SELECT
        T.SeatNo
      FROM 
        Transaction as T
      WHERE
        T.FlightId = FlightId AND
        T.PlaneId = PlaneId AND
        (T.TransType = 1 OR 
         (T.TransType = 0 AND NOT IsReservationExpired(T.FlightId, T.SeatNo)));
END;
$BODY$ LANGUAGE plpgsql;

\end{lstlisting}

2. \textbf{Reserve(FlightId, SeatNo)~--- пытается забронировать место. Возвращает истину, если удалось и ложь — в противном случае.}

Если старая бронь истекла, удаляет ее из $Transactions$. В случае, если бронирование успешно, добавляет в таблицу бронь.

\begin{lstlisting}[
           language=SQL,
           showspaces=false,
           basicstyle=\ttfamily,
           numbers=none,
           numberstyle=\tiny,
           commentstyle=\color{gray}
        ]
    CREATE FUNCTION Reserve(FlightId INT, SeatNo INT)
    RETURNS BOOLEAN AS $BODY$
      DECLARE FlightTime TIMESTAMP;
      DECLARE PlaneId INT;
      DECLARE IsClosedByRequest BOOLEAN DEFAULT FALSE;
      DECLARE TransTime TIMESTAMP DEFAULT NULL;
      DECLARE TransType INT DEFAULT NULL;
    BEGIN
      SELECT (F.FlightTime, F.isClosedByRequest) INTO
        FlightTime, IsClosedByRequest
      FROM
        Flights as F
      WHERE
        F.FlightId = FlightId;

      IF NOT IsClosedByRequest AND NOW() <= FlightTime - INTERVAL '24 hours' THEN
        SELECT (T.PlaneId, T.TransTime, T.TransType) INTO
          PlaneId, TransTime, TransType
        FROM
          Transaction as T
        WHERE
          T.FlightId = FlightId AND
          T.SeatNo = SeatNo;
        IF (TransTime IS NULL AND
            TransType IS NULL) OR
           (TransType = 0 AND
            IsReservationExpired(FlightId, SeatNo)) THEN
          DELETE FROM
            Transaction as T
          WHERE
            T.FlightId = FlightId AND
            T.SeatNo = SeatNo;

          INSERT INTO Transaction
            (FlightId, PlaneId, SeatNo, TransTime, TransType)
          VALUES
            (FlightId, PlaneId, SeatNo, now(), 0);
          RETURN TRUE;
        ELSE
          //Seat is already reserved (maybe by yourself, but it doesn't matter)
          RETURN FALSE;
        END IF;
      ELSE
        //Reserving is closed or it is too late
        RETURN FALSE;
      END IF;
    END;
    $BODY$ LANGUAGE plpgsql;
\end{lstlisting}

3. \textbf{ExtendReservation(FlightId, SeatNo) — пытается продлить бронь места. Возвращает истину, если удалось и ложь — в противном случае.}

Если бронь истекла, продлить ее нельзя, она удаляется из таблицы $Transactions$. Иначе, обновляется, устанавливается $TransTime = now()$.

\begin{lstlisting}[
           language=SQL,
           showspaces=false,
           basicstyle=\ttfamily,
           numbers=none,
           numberstyle=\tiny,
           commentstyle=\color{gray}
        ]
    CREATE FUNCTION ExtendReservation(FlightId INT, SeatNo INT)
    RETURNS BOOLEAN AS $BODY$
      DECLARE PlaneId INT;
    BEGIN
      IF IsReservationExpired(FlightId, SeatNo) THEN
        //Reservation is expired, we can't update it.
        DELETE FROM
          Transaction as T
        WHERE
          T.FlightId = FlightId AND
          T.SeatNo = SeatNo AND
          T.TransType = 0;
        RETURN FALSE;
      ELSE
        UPDATE
          Transaction as T
        SET 
          T.TransTime = now()
        WHERE
          T.FlightId = FlightId AND
          T.SeatNo = SeatNo AND
          T.TransType = 0;
        RETURN TRUE;
      END IF;
    END;
    $BODY$ LANGUAGE plpgsql;
\end{lstlisting}

4. \textbf{BuyFree(FlightId, SeatNo) — пытается купить свободное место. Возвращает истину, если удалось и ложь — в противном случае.}

\begin{lstlisting}[
           language=SQL,
           showspaces=false,
           basicstyle=\ttfamily,
           numbers=none,
           numberstyle=\tiny,
           commentstyle=\color{gray}
        ]
    CREATE FUNCTION BuyFree(FlightId INT, SeatNo INT)
    RETURNS BOOLEAN AS $BODY$
      DECLARE PlaneId INT;
      DECLARE FlightTime TIMESTAMP;
      DECLARE IsClosedByRequest BOOLEAN DEFAULT FALSE;
      DECLARE TransTime TIMESTAMP DEFAULT NULL;
      DECLARE TransType INT DEFAULT NULL;
    BEGIN
      SELECT (F.FlightTime, F.isClosedByRequest) INTO
        FlightTime, IsClosedByRequest
      FROM
        Flights as F
      WHERE
        F.FlightId = FlightId;

      IF NOT IsClosedByRequest AND NOW() <= FlightTime - INTERVAL '2 hours' THEN
        SELECT (T.PlaneId, T.TransTime, T.TransType) INTO
          PlaneId, TransTime, TransType
        FROM
          Transaction as T
        WHERE
          T.FlightId = FlightId AND
          T.SeatNo = SeatNo;
        IF TransTime IS NULL AND TransType IS NULL THEN
          //Seat is free
          INSERT INTO Transaction
            (FlightId, PlaneId, SeatNo, TransTime, TransType)
          VALUES
            (FlightId, PlaneId, SeatNo, now(), 1);
          RETURN TRUE;
        ELSE
          //Seat it already bought or reserved, can't buy it
          RETURN FALSE;
        END IF;
      ELSE
        //Bying is closed or it is too late
        RETURN FALSE;
      END IF;
    END;
    $BODY$ LANGUAGE plpgsql;
\end{lstlisting}

5. \textbf{BuyReserved(FlightId, SeatNo) — пытается выкупить забронированное место. Возвращает истину, если удалось и ложь — в противном случае.}

\begin{lstlisting}[
           language=SQL,
           showspaces=false,
           basicstyle=\ttfamily,
           numbers=none,
           numberstyle=\tiny,
           commentstyle=\color{gray}
        ]
    CREATE FUNCTION BuyReserved(FlightId INT, SeatNo INT)
    RETURNS BOOLEAN AS $BODY$
      DECLARE PlaneId INT;
      DECLARE FlightTime TIMESTAMP;
      DECLARE IsClosedByRequest BOOLEAN DEFAULT FALSE;
      DECLARE TransTime TIMESTAMP DEFAULT NULL;
      DECLARE TransType INT DEFAULT NULL;
    BEGIN
      SELECT (F.FlightTime, F.isClosedByRequest) INTO
        FlightTime, IsClosedByRequest
      FROM
        Flights as F
      WHERE
        F.FlightId = FlightId;

      IF NOT IsClosedByRequest AND NOW() <= FlightTime - INTERVAL '2 hours' THEN
        SELECT (T.PlaneId, T.TransTime, T.TransType) INTO
          PlaneId, TransTime, TransType
        FROM
          Transaction as T
        WHERE
          T.FlightId = FlightId AND
          T.SeatNo = SeatNo;
        IF TransType = 0 AND NOT IsReservationExpired(FlightId, SeatNo) THEN
          //Seat is reserved
          //Here we need to check that it was reversed by ourselves
          //But we have no information about reservation in our tables
          //In 8th hometask I used passport series/no to check info, but here
          //we haven't them in parameters, so we'll just leave this step,
          //assuming that reservation was done by ourselves.
          DELETE FROM
            Transaction as T
          WHERE
            T.FlightId = FlightId AND
            T.SeatNo = SeatNo;

          INSERT INTO Transaction
            (FlightId, PlaneId, SeatNo, TransTime, TransType)
          VALUES
            (FlightId, PlaneId, SeatNo, now(), 1);
          RETURN TRUE;
        ELSE
          //Seat is already bought or reserved, can't buy it
          RETURN FALSE;
        END IF;
      ELSE
        //Bying is closed or it's too late
        RETURN FALSE;
      END IF;
    END;
    $BODY$ LANGUAGE plpgsql;
\end{lstlisting}

6. \textbf{FlightStatistics() — возвращает статистику по рейсам: возможность бронирования и покупки, число свободных, забронированных и проданных мест.}

\begin{lstlisting}[
           language=SQL,
           showspaces=false,
           basicstyle=\ttfamily,
           numbers=none,
           numberstyle=\tiny,
           commentstyle=\color{gray}
        ]
    CREATE FUNCTION MayBuy(FId INT)
    RETURNS BOOLEAN AS $BODY$
      DECLARE PId INT;
      DECLARE IsClosedByRequest BOOLEAN DEFAULT FALSE;
      DECLARE FTime TIMESTAMP DEFAULT NULL;
      DECLARE FreeSeatsCount INT DEFAULT NULL;
    BEGIN
      SELECT (F.IsClosedByRequest, F.FTime, F.PlaneId)
      INTO IsClosedByRequest, FTime, PId
      FROM Flights F
      WHERE F.FlightId = FId;
      
      SELECT COUNT(*)
      INTO FreeSeatsCount
      FROM FreeSeats(FId);
      IF FreeSeatsCount == 0 OR 
        IsClosedByRequest OR NOW() > FTime - INTERVAL '2 hours' THEN
        RETURN FALSE;
      ELSE
        RETURN TRUE;
      END IF;
    END;
    $BODY$ LANGUAGE plpgsql;

    CREATE FUNCTION MayReserve(FId INT)
    RETURNS BOOLEAN AS $BODY$
      DECLARE PId INT;
      DECLARE IsClosedByRequest BOOLEAN DEFAULT FALSE;
      DECLARE FTime TIMESTAMP DEFAULT NULL;
      DECLARE FreeSeatsCount INT DEFAULT NULL;
    BEGIN
      SELECT (F.IsClosedByRequest, F.FTime, F.PlaneId)
      INTO IsClosedByRequest, FTime, PId
      FROM Flights F
      WHERE F.FlightId = FId;
      
      SELECT COUNT(*)
      INTO FreeSeatsCount
      FROM FreeSeats(FId);
      IF FreeSeatsCount == 0 OR 
        IsClosedByRequest OR NOW() > FTime - INTERVAL '24 hours' THEN
        RETURN FALSE;
      ELSE
        RETURN TRUE;
      END IF;
    END;
    $BODY$ LANGUAGE plpgsql;

    CREATE FUNCTION FreeSeatsCount(FId INT)
    RETURNS INT AS $BODY$
      DECLARE FreeSeatsCount INT DEFAULT NULL;
    BEGIN
      SELECT COUNT(*)
      INTO FreeSeatsCount
      FROM FreeSeats(FId);
      RETURN FreeSeatsCount;
    END;
    $BODY$ LANGUAGE plpgsql;

    CREATE FUNCTION ReservedSeatsCount(FId INT)
    RETURNS INT AS $BODY$
      DECLARE ReservedSeatsCount INT DEFAULT NULL;
    BEGIN
      SELECT COUNT(*)
      INTO ReservedSeatsCount
      FROM (
        SELECT
          T.SeatNo
        FROM
          Transaction as T
        WHERE
          T.FlightId = FId AND
          T.TransType = 0
      ) as SUBQ;
      RETURN ReservedSeatsCount;
    END;
    $BODY$ LANGUAGE plpgsql;

    CREATE FUNCTION BoughtSeatsCount(FId INT)
    RETURNS INT AS $BODY$
      DECLARE BoughtSeatsCount INT DEFAULT NULL;
    BEGIN
      SELECT COUNT(*)
      INTO BoughtSeatsCount
      FROM (
        SELECT
          T.SeatNo
        FROM
          Transaction as T
        WHERE
          T.FlightId = FId AND
          T.TransType = 1
      ) as SUBQ;
      RETURN BoughtSeatsCount;
    END;
    $BODY$ LANGUAGE plpgsql;

    CREATE FUNCTION FlightStatistics()
    RETURNS TABLE(FlightId INT, MayBuy BOOLEAN, MayReserve BOOLEAN, 
                  FreeSeats INT, ReservedSeats INT, BoughtSeats INT) AS $BODY$
    BEGIN
      RETURN QUERY
      SELECT F.FlightId, 
            MayBuy(F.FlightId) as MayBuy,
            MayReserve(F.FlightId) as MayReserve,
            FreeSeatsCount(F.FlightId) as FreeSeats,
            ReservedSeatsCount(F.FlightId) as ReservedSeats,
            BoughtSeatsCount(F.FlightId) as BoughtSeats
      FROM Flights F;
    END;
    $BODY$ LANGUAGE plpgsql;
\end{lstlisting}

\end{document}