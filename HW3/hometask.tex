\documentclass[11pt,a4paper,oneside]{article}

\usepackage[english,russian]{babel}
\usepackage[T2A]{fontenc}
\usepackage[utf8]{inputenc}
\usepackage[russian]{olymp}
\usepackage{graphicx}
\usepackage{expdlist}
\usepackage{mfpic}
\usepackage{amsmath}
\usepackage{amssymb}
\usepackage{comment}
%\usepackage{listings}
\usepackage{epigraph}
\usepackage{url}
\usepackage{ulem}

\DeclareMathOperator{\nott}{not}

\begin{document}

\renewcommand{\t}[1]{\mbox{\texttt{#1}}}
\newcommand{\s}[1]{\mbox{``\t{#1}''}}
\newcommand{\eps}{\varepsilon}
\renewcommand{\phi}{\varphi}
\newcommand{\plainhat}{{\char 94}}

\newcommand{\Z}{\mathbb{Z}}
\newcommand{\w}[1]{``\t{#1}''}

\binoppenalty=10000
\relpenalty=10000

\createsection{\Note}{Комментарий}

\contest{Домашнее задание по базам данных}{Филиппов Дмитрий, М3439}{24 октября 2016 года}

Филиппов Дмитрий, М3439
\newline

\begin{LARGE} \textbf{Домашнее задание 3.} \end{LARGE}
\newline

\textbf{Множество атрибутов:} StudentId, StudentName, GroupId, GroupName, CourseId, CourseName, LecturerId, LecturerName, Mark.
\newline

\textbf{Функциональные зависимости:}
\begin{itemize}
\item StudentId $\rightarrow$ StudentName
\item GroupId $\rightarrow$ GroupName
\item CourseId $\rightarrow$ CourseName
\item LecturerId $\rightarrow$ LecturerName
\item StudentId, CourseId $\rightarrow$ LecturerId, GroupId, Mark
\end{itemize}

\textbf{Возможные множества ключей:}
\begin{itemize}
\item $\{$ StudentId, CourseId $\}$, потому что $\{$ StudentId, CourseId $\}^+$ является полным множеством атрибутов, и меньше множества не существует.
\end{itemize}

\textbf{Неприводимое множество для данного множества функциональных зависимостей:}

Расщепим правые части:

\begin{itemize}
\item StudentId $\rightarrow$ StudentName
\item GroupId $\rightarrow$ GroupName
\item CourseId $\rightarrow$ CourseName
\item LecturerId $\rightarrow$ LecturerName
\item StudentId, CourseId $\rightarrow$ LecturerId
\item StudentId, CourseId $\rightarrow$ GroupId
\item StudentId, CourseId $\rightarrow$ Mark
\end{itemize}

Переходим к процедуре удаления по одному атрибуту из левых частей:

\begin{itemize}
\item StudentId $\rightarrow$ StudentName
\item GroupId $\rightarrow$ GroupName
\item CourseId $\rightarrow$ CourseName
\item LecturerId $\rightarrow$ LecturerName
\item StudentId, CourseId $\rightarrow$ LecturerId
\item StudentId, CourseId $\rightarrow$ Mark
\item StudentId $\rightarrow$ GroupId
\end{itemize}

Ненужных зависимостей тут нет. Значит получили неприводимое множество функциональных зависимостей.
\newline

\begin{LARGE} \textbf{Домашнее задание 4.} \end{LARGE}
\newline

\textbf{Приведение в 5НФ:}

\textit{1НФ}: \underline{StudentId}, StudentName, GroupId, GroupName, \underline{CourseId}, CourseName, LecturerId, LecturerName, Mark.

\textit{2НФ}: 
\begin{itemize}
\item \underline{StudentId}, StudentName~--- StudentName зависит только от StudentId
\item \underline{StudentId}, GroupId, GroupName~--- GroupId зависит только от StudentId, GroupName~--- от GroupId, с которым все ок
\item \underline{CourseId}, CourseName~--- CourseName зависит только от CourseId
\item \underline{StudentId}, \underline{CourseId}, LecturerId, LecturerName, Mark~--- LecturerId зависит от StudentId, CourseId; LecturerName~--- от LecturerId; Mark~--- от StudentId, CourseId
\end{itemize}

\textit{3НФ}:
\begin{itemize}
\item \underline{StudentId}, StudentName
\item \underline{StudentId}, GroupId
\item \underline{GroupId}, GroupName
\item \underline{CourseId}, CourseName
\item \underline{LecturerId}, LecturerName
\item \underline{StudentId}, \underline{CourseId}, LecturerId, Mark
\end{itemize}

\textit{4НФ} и \textit{5НФ} выглядят так же, как и \textit{3НФ}.

\end{document}